%% start of file `my-CV-2013.tex'.
%% Copyright 2006-2013 Xavier Danaux (xdanaux@gmail.com).
%
% This work may be distributed and/or modified under the
% conditions of the LaTeX Project Public License version 1.3c,
% available at http://www.latex-project.org/lppl/.

% possible options include font size ('10pt', '11pt' and '12pt'),
% paper size ('a4paper', 'letterpaper', 'a5paper', 'legalpaper',
% 'executivepaper' and 'landscape') and font family ('sans' and 'roman')
\documentclass[11pt,a4paper,roman]{moderncv} 

% moderncv themes
% style options are 'casual' (default), 'classic', 'oldstyle' and 'banking'
\moderncvstyle{classic}

% color options 'blue' (default), 'orange', 'green',
% 'red', 'purple', 'grey' and 'black'
% \moderncvcolor{orange}

\usepackage{happycv}

% to set the default font; use '\sfdefault' for the default sans serif font,
% '\rmdefault' for the default roman one, or any tex font name
% \renewcommand{\familydefault}{\sfdefault}

% uncomment to suppress automatic page numbering for CVs longer than one page
%\nopagenumbers{}

% character encoding
% if you are not using xelatex ou lualatex,
% replace by the encoding you are using
\usepackage{DejaVuSansMono}
\usepackage[utf8]{inputenc}
\usepackage{fontspec,indentfirst}
\usepackage{xunicode}% provides unicode character macros
\usepackage{xltxtra} % provides some fixes/extras
\usepackage{wasysym}

% if you need to use CJK to typeset your resume in Chinese, Japanese or Korean
% \usepackage{CJKutf8}
\usepackage{tikz}
\usepackage{footmisc}

\XeTeXlinebreaklocale "zh"
\XeTeXlinebreakskip = 0pt plus 1pt minus 0.1pt

\newfontfamily\xingkai{"华文行楷"}
\newfontfamily\caiyun{"华文彩云"}
\newfontfamily\kai{"楷体"}
\newfontfamily\fs{"仿宋"}
\newfontfamily\li{"隶书"}
\newfontfamily\xinwei{"华文新魏"}
\newfontfamily\yao{"方正姚体"}
\newfontfamily\hei{"黑体"}
\newfontfamily\song{"新宋体"}
\newfontfamily\mshei{"微软雅黑"}

\setmainfont{"宋体"}

\renewcommand{\baselinestretch}{1.1}
\newenvironment{tightitemize}
{\begin{itemize}\setlength{\parskip}{0pt}}
{\end{itemize}}
% adjust the page margins
\usepackage[scale=0.8]{geometry}

% if you want to change the width of the column with the dates
%\setlength{\hintscolumnwidth}{3cm}
% for the 'classic' style, if you want to force the width allocated to
% your name and avoid line breaks. be careful though, the length is normally
% calculated to avoid any overlap with your personal info;
% use this at your own typographical risks...
%\setlength{\makecvtitlenamewidth}{10cm}

\makeatletter
\tikzset{
    tl@startyear/.append style={
        xshift=(0.5-\tl@startfraction)*\hintscolumnwidth,
        anchor=base
    }
}
\makeatother

% quote, optional, remove / comment the line if not wanted.
% \myquote{MAKE A LITTLE SPACE, MAKE A BETTER PLACE.}{Michael Jackson: <<Heal The World>>}

% to show numerical labels in the bibliography (default is to show no labels);
% only useful if you make citations in your resume
%\makeatletter
%\renewcommand*{\bibliographyitemlabel}{\@biblabel{\arabic{enumiv}}}
%\makeatother

% CONSIDER REPLACING THE ABOVE BY THIS
%\renewcommand*{\bibliographyitemlabel}{[\arabic{enumiv}]}

% bibliography with mutiple entries
%\usepackage{multibib}
%\newcites{book,misc}{{Books},{Others}}
%-------------------------------------------------------------------------------
%                                 content
%-------------------------------------------------------------------------------
\begin{document}
% \begin{CJK*}{UTF8}{gbsn} % to typeset your resume in Chinese using CJK
\makecvtitle

% footnote in my moderncv.
\footnotetext[1]{http://hbase.apache.org/}
\footnotetext[2]{https://code.google.com/p/gperftools/}
\footnotetext[3]{www.neo4j.org}
\footnotetext[4]{http://zookeeper.apache.org/}


\section{\li{教育经历}}
\tlcventry{2011}{0}{\hei{硕士在读}}{网络数据科学与技术重点实验室}{中国科学院计算技术研究所}{北京}{}
\tllabelcventry{2007}{2011}{2007-2011}{\hei{学士}}{计算机科学与技术}{华中科技大学}{武汉, 年级排名 5\% 以内}{}

\section{\li{专业技能}}
\cvdoubleitem{操作系统:}{Ubuntu, Fedora, CentOS}{数据库:}{MySQL, HBase\footnotemark[1], Memcached}
\cvdoubleitem{编程语言:}{C/C++, Java, Python, Bash, Lua}{工具链:}{GCC, GDB, Binutils, LLVM/Clang}
\cvdoubleitem{代码工具:}{Vim, Git, SVN, CMake, Eclipse}{其他工具:}{Valgrind,SystemTap,Gperftools\footnotemark[2]}
\cvdoubleitem{文档编写:}{LaTeX, Markdown, Word, PPT}{外语水平:}{英语~6~级(549)}
%\cvdoubleitem{个人博客}{\textcolor[rgb]{0.4,0.4,0.4}{\texttt{\httplink{www.cnblogs.com/haippy/}}}}{\textsc{Github} 主页}{\textcolor[rgb]{0.4,0.4,0.4}{\texttt{\httplink{www.github.com/forhappy/}}}}

\section{\li{硕士期间参与项目}}
\tllabelcventry{2012}{0}{2013.3-2013.8}{\hei{网络图搜索项目开发}}{主要开发人员}{编程语言: Java}{}{
\begin{tightitemize}
	\item 关键技术:~HBase, Neo4J\footnotemark[3], RESTful API, Zookeeper\footnotemark[4].
	\item 利用 HBase 保存图数据。
	\item 利用 Neo4J 缓存 HBase 保存的图数据,加速关系的计算。
	\item 为前台用户设计了一套 RESTful 接口。
\end{tightitemize}}
\tllabelcventry{2012}{2013}{2012.9-2013.4}{\hei{某网络搜索项目维护与二期开发}}{主要开发人员}{编程语言:~C++, Java}{}{
\begin{tightitemize}
	\item 关键技术:~Thrift, NLPbamboo, Neo4J, RESTful API, Libevent, Zookeeper.
	\item 编写和更新了 Neo4J 图数据库的数据入库程序与查询接口.
	\item 实现了一个简单的微博寻人的工具,  通过微博搜索页面 s.weibo.com 爬取微博用户的个人信息.
\end{tightitemize}}
\tllabelcventry{2012}{0}{2012.12-至今}{\hei{某新媒体项目的网络视频地址解析库~\textsc{Android}~平台移植}}{主要开发人员}{编程语言:~\textsc{Java}}{}{
\begin{tightitemize}
	\item 关键技术:~HTTP, 协议分析.
\end{tightitemize}}
\tllabelcventry{2011}{2012}{2012.3-2012.6}{\hei{\textsc{HBase}~数据导入工具的设计与实现}}{设计与实现人员}{编程语言:~\textsc{Java}}{}{
\begin{tightitemize}
	\item 关键技术:~HBase, RDBMS(MySQL, Oracle), JDBC, SQL, Sqoop.
	\item 基本介绍:~Haiep 功能上与 apache sqoop 类似, 用于从关系数据库向 HBase 导入数据.
	\item 项目主页:~\textcolor[rgb]{0.4,0.4,0.4}{\texttt{\httplink{forhappy.github.com/haiep/}}}
\end{tightitemize}}

\footnotetext[5]{有效代码, 下同.}
\footnotetext[6]{计算机技术与软件专业技术资格(水平)考试}

\section{\li{个人开源项目}}
% \tlcventry{2007--2011}{Degree}{Institution}{City}{\textit{Grade}}{Description}
\tllabelcventry{2011}{0}{2011.9}{\texttt{\textcolor[rgb]{0.55,0,0}{cpy-leveldb}}}{\textsc{C\&Python}}{1.5K SLOC\footnotemark[5]}{\textcolor[rgb]{0.4,0.4,0.4}{\texttt{\httplink{www.github.com/forhappy/cpy-leveldb}}}}{
\begin{tightitemize}
	\item 基本介绍:~\textsc{Google Leveldb}~的~\textsc{Python}~绑定, 基于~\textsc{Leveldb~C~API}.
\end{tightitemize}}
\tllabelcventry{2012}{0}{2012.9}{\texttt{\textcolor[rgb]{0.55,0,0}{ossc}}}{\textsc{C\&Shell}}{18K SLOC}{\textcolor[rgb]{0.4,0.4,0.4}{\texttt{\httplink{www.github.com/forhappy/OSSC}}}}{
\begin{tightitemize}
	\item 基本介绍:~阿里云开放存储服务(Open Storage Service: OSS)~C~SDK, 首届阿里云开发者大赛获奖作品(最佳实用奖: 奖金~5~万人民币整, 合作者:~王维.)
\end{tightitemize}}
\tllabelcventry{2012}{0}{2012.12}{\texttt{\textcolor[rgb]{0.55,0,0}{reveldb}}}{\textsc{C\&Shell}}{12K SLOC}{\textcolor[rgb]{0.4,0.4,0.4}{\texttt{\httplink{www.github.com/forhappy/reveldb}}}}{
\begin{tightitemize}
	\item 基本介绍:~高性能~\textsc{Key-Value}~数据库服务器, 支持插件式存储引擎及~\textsc{Restful}~访问协议.
\end{tightitemize}}
\tllabelcventry{2013}{0}{2013.1}{\texttt{\textcolor[rgb]{0.55,0,0}{neo4cpp}}}{\textsc{C++}}{3K SLOC}{\textcolor[rgb]{0.4,0.4,0.4}{\texttt{\httplink{www.github.com/forhappy/neo4j-cpp-driver}}}}{
\begin{tightitemize}
	\item 基本介绍:~开源图数据库~Neo4J~的~C++~客户端(官方未提供~C++~客户端).
\end{tightitemize}}
\tllabelcventry{2013}{0}{2013.3}{\texttt{\textcolor[rgb]{0.55,0,0}{uvbook}}}{\textrm{reStructuredText}}{}{\textcolor[rgb]{0.4,0.4,0.4}{\texttt{\httplink{www.github.com/forhappy/uvbook}}}}{
\begin{tightitemize}
	\item 基本介绍:~\textsc{Libuv} 编程指南中文翻译.
\end{tightitemize}}
\tllabelcventry{2013}{0}{2013.4}{\texttt{\textcolor[rgb]{0.55,0,0}{zklua}}}{\textsc{C\&Lua}}{3.5K SLOC}{\textcolor[rgb]{0.4,0.4,0.4}{\texttt{\httplink{www.github.com/forhappy/zklua}}}}{
\begin{tightitemize}
	\item 基本介绍:~\textsc{Apache Zookeeper} 的 Lua 客户端, 基于 \textsc{Zooeeper C API}.
\end{tightitemize}}
\tllabelcventry{2013}{0}{2013.4}{\texttt{\textcolor[rgb]{0.55,0,0}{lua-snappy}}}{\textsc{C\&Lua}}{200 SLOC}{\textcolor[rgb]{0.4,0.4,0.4}{\texttt{\httplink{https://github.com/forhappy/lua-snappy}}}}{
\begin{tightitemize}
	\item \textcolor[rgb]{0,0,0.37}{Introduction}:~\textsc{Google 压缩库 Snappy 的 Lua 接口绑定.}
\end{tightitemize}}

\section{\li{本科及硕士阶段所获主要奖项和证书}}
\subsection{主要奖项}
\cvitem{2012.12}{\kai{中科院计算技术研究所腾讯优秀奖学金}}
\cvitem{2012.11}{\kai{首届阿里云开发者大赛最佳实用奖}}
\cvitem{2011.07}{\kai{华中科技大学优秀毕业生}}
\cvitem{2011.07}{\kai{湖北省学士学位优秀毕业论文}}
\cvitem{2010.11}{\kai{华中科技大学学习优秀奖学金}}
\cvitem{2009.11}{\kai{国家励志奖学金}}
\cvitem{2008.11}{\kai{国家励志奖学金}}

\subsection{证书}
\cvitem{2010.10}{\kai{系统分析师(高级)\footnotemark[6]}}
\cvitem{2010.03}{\kai{软件设计师(中级)\footnotemark[6]}}

\section{\li{自我评价}}
\cvitem{}{{为人随和, 责任心强, 喜欢跑步, 登山, 健身, 热爱代码与数学\smiley{}}}

% if you are typesetting your resume in Chinese using CJK;
% the \clearpage is required for fancyhdr to work correctly with CJK,
% though it kills the page numbering by making \lastpage undefined
\clearpage
% \end{CJK*}
\end{document}

%% end of file `my-CV-2013.tex'.
